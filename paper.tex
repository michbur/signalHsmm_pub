\documentclass[fleqn,10pt,twoside]{gcb15submission}
\usepackage{url}\urlstyle{same}


\title{signalHsmm - a novel semi-Markov model of signal peptides}

\author[1]{Micha\l{}  Burdukiewicz}
\author[2]{Piotr Sobczyk}
\author[1]{Pawe\l{} B\l{}a\.{z}ej}
\author[1]{Pawe\l{} Mackiewicz}
\affil[1]{University of Wroc\l{}aw, Department of Genomics, Poland}
\affil[2]{Wroc\l{}aw University of Technology, Department of Mathematics, Poland}

\keywords{Keyword1, Keyword2, Keyword3}

\begin{abstract}
Dummy abstract text.
Please do not use any commands or any formatting within the abstract.
Also, please do not use subsections, etc. within the abstract.
\end{abstract}

\begin{document}
\flushbottom
\maketitle
\thispagestyle{empty}


\section*{Introduction}

Your introduction goes here.
Some examples of commonly used {\LaTeX} commands and features are listed below, to help you get started.

All \textbf{accepted} submissions will be published as a \textbf{preprint collection} with \textbf{PeerJ} (\url{http://www.peerj.com}).
It will be clearly visible that the work
\begin{enumerate}
\item is a preprint (although reviewed and accepted by the GCB program committee)
\item has been presented at GCB'15
\end{enumerate}
by including a special GCB logo in the final version

If you decide to submit a full version after GCB to PeerJ or to PeerJ Computer Science, the procedure is very easy.
Of course, you are free to submit your paper to any other journal that allows existing preprints.


\subsection*{About PeerJ}

PeerJ is an award-winning open access publisher covering both computer science and the biological and medical sciences. 
PeerJ provides authors with three publication venues: PeerJ and PeerJ Computer Science (peer-reviewed academic journals) and PeerJ PrePrints (a 'pre-print server'). See \url{https://peerj.com/about/publications/} for more information.

The PeerJ model allows an author to publish articles in their peer-reviewed journal via the purchase of a lifetime Publication Plan. Prices start from just \$99 (a one-off payment) which entitles an author to the lifetime ability to publish 1 article per year for free.
The fee applies to all authors of an article, which encourages a reasonable number of co-authors and is generally less than for other open-access publishers.


\section*{Some \LaTeX{} Examples}
\label{sec:examples}

Use section and subsection commands to organize your document. \LaTeX{} handles all the formatting and numbering automatically. Use ref and label commands for cross-references.

\subsection*{Figures and Tables}

Use the table environment and the tabular command for basic tables --- see Table~\ref{tab:widgets}, for example.

To include a figure in your document, use the figure environment and the includegraphics command as in the code for Figure~\ref{fig:uaruhr}.

\begin{figure}[ht]\centering
\includegraphics[width=0.6\textwidth]{uaruhr}
\caption{An example image.}
\label{fig:uaruhr}
\end{figure}

\begin{table}[ht]
\centering
\begin{tabular}{l|r}
Item & Quantity \\\hline
Widgets & 42 \\
Gadgets & 13
\end{tabular}
\caption{\label{tab:widgets}An example table.}
\end{table}

\subsection*{Citations}

LaTeX formats citations and references automatically using the bibliography records in your .bib file.
Use the \verb|\cite| command for an inline citation, like \cite{MuellerRahmann+2003RobustEstimation}, and the \verb|\citep| command for a citation in parentheses \citep{MuellerRahmann+2003RobustEstimation}.

\subsection*{Mathematics}

\LaTeX{} is great at typesetting mathematics. Let $X_1, X_2, \ldots, X_n$ be a sequence of independent and identically distributed random variables with $\text{E}[X_i] = \mu$ and $\text{Var}[X_i] = \sigma^2 < \infty$, and let
$$S_n = \frac{X_1 + X_2 + \cdots + X_n}{n}
      = \frac{1}{n}\sum_{i}^{n} X_i$$
denote their mean. Then as $n$ approaches infinity, the random variables $\sqrt{n}(S_n - \mu)$ converge in distribution to a normal $\mathcal{N}(0, \sigma^2)$.

\subsection*{Lists}

You can make lists with automatic numbering \dots

\begin{enumerate}[noitemsep] 
\item Like this,
\item and like this.
\end{enumerate}
\dots or bullet points \dots
\begin{itemize}[noitemsep] 
\item Like this,
\item and like this.
\end{itemize}
\dots or with words and descriptions \dots
\begin{description}
\item[Word] Definition
\item[Concept] Explanation
\item[Idea] Text
\end{description}



\section*{Methods}

In a bioinformatics paper, the methods section should be the most important one.
Therefore, feel free to have more than one method section or to choose a more meaningful title for it.

\subsection*{Subsection}

Here is an interesting equation that may be helpful in some situations:
\begin{equation}
\cos^3 \theta =\frac{1}{4}\cos\theta+\frac{3}{4}\cos 3\theta
\label{eq:refname2}
\end{equation}

\paragraph{Paragraph}
Nothing to see here. Move on.

\paragraph{Paragraph}
Really.
See Figure~\ref{fig:results} for more interesting results.

\begin{figure}[ht]\centering
\includegraphics[width=0.6\textwidth]{results}
\caption{Can you guess which function this is?}
\label{fig:results}
\end{figure}



\section*{Results and Discussion}

You may want to separate results, discussion and conclusion, according to your needs.

Please submit the final pdf file via EasyChair to the GCB'15 program committee by June 30, 2015. 


\section*{Acknowledgments}
Thank you for your support!


\bibliography{sample}

\end{document}
